\begin{figure}[H]
  \centering
  \begin{tikzpicture}[
    node distance=2.2cm,
    font=\scriptsize,
    >=Latex,
    every node/.style={rectangle, draw=black!35, fill=black!2, rounded corners=6pt, align=center, minimum width=3.4cm, minimum height=1.1cm},
    line/.style={->, thick, draw=black!35}
  ]
    \node (bedeutung) at (0,2.2) {Bedeutung\\(FU$_1$)};
    \node (effekte) at (0,0) {Effekte\\(FU$_{2a}$/FU$_{2b}$)};
    \node (faktoren) at (0,-2.2) {Effektfaktoren};

    \node (moglichkeiten) at (4.4,2.2) {Möglichkeiten\\(FU$_5$)};
    \node (mechanismen) at (4.4,0) {Mechanismen\\(FU$_{4a}$/FU$_{4b}$)};
    \node (konzeption) at (4.4,-2.2) {Konzeption\\(FU$_3$)};

    \node (kompetenzen) at (8.8,2.2) {Kompetenzen\\(FU$_6$)};
    \node (kausal) at (8.8,0) {Kausalgesetze\\(FU$_7$)};
    \node (wirkgefuge) at (8.8,-2.2) {Wirkgefüge};

    \draw[line] (bedeutung) -- (effekte);
    \draw[line] (effekte) -- (faktoren);
    \draw[line] (moglichkeiten) -- (mechanismen);
    \draw[line] (mechanismen) -- (konzeption);
    \draw[line] (kompetenzen) -- (kausal);
    \draw[line] (kausal) -- (wirkgefuge);

    \draw[line] (bedeutung) -- (moglichkeiten);
    \draw[line] (effekte) -- (mechanismen);
    \draw[line] (faktoren) -- (konzeption);
    \draw[line] (moglichkeiten) -- (kompetenzen);
    \draw[line] (mechanismen) -- (kausal);
    \draw[line] (konzeption) -- (wirkgefuge);
  \end{tikzpicture}
  \captionsetup{justification=justified,singlelinecheck=false}
  \caption{Abfolge der Forschungsunterfragen: von Bedeutung und Effekten über Mechanismen und Konzeption hin zu Kompetenzen, Kausalgesetzen und Wirkgefüge.}
  \caption*{\footnotesize Visualisiert ist die Sequenz der Forschungsunterfragen (FU$_1$–FU$_7$) als gerichteter Pfad mit Rückbezügen: Bedeutung $\rightarrow$ Effekte $\rightarrow$ Effektfaktoren; flankiert von Möglichkeiten/Mechanismen/Konzeption sowie Kompetenz- und Kausalannahmen bis zum Wirkgefüge.}
  \label{fig:fu-sequenz}
\end{figure}

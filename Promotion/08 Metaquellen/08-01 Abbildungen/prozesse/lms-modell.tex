\begin{figure}[H]
  \centering
  \resizebox{\linewidth}{!}{%
    \begin{tikzpicture}[
      node distance=1.4cm and 2.0cm,
      font=\scriptsize,
      >=Latex,
      process/.style={rectangle, draw=black!35, fill=black!2, rounded corners=6pt, align=center, minimum width=3.4cm, minimum height=1.1cm},
      line/.style={->, thick, draw=black!35},
      dashedline/.style={->, thick, draw=black!35, dashed}
    ]
      \node[process] (curriculum) {Handlungssituationen\\(32 curriculare Einheiten)};
      \node[process, right=of curriculum, xshift=1.4cm] (lms) {LMS-Kern\\Kurse, Ressourcen, Aufgaben, Feedback};
      \node[process, right=of lms, xshift=1.4cm] (cohort) {Ausbildungskurse\\(3 Kohortenräume)};
      \node[process, above=of lms, yshift=0.6cm] (content) {Content\\Fachliteratur, Medien};
      \node[process, below=of lms, yshift=-0.6cm] (lernorte) {Lernorte\\Schule, Lehrrettungswache, Krankenhaus};

      \draw[line] (curriculum) -- node[above, align=center]{curriculare Struktur} (lms);
      \draw[line] (lms) -- node[above, align=center]{Umsetzung \& Steuerung} (cohort);
      \draw[line] (curriculum) to[bend left=22] node[pos=0.55, above, yshift=4pt]{Aufgaben, Standards} (cohort);
      \draw[line] (cohort) to[bend left=22] node[pos=0.45, below, yshift=-4pt]{Erkenntnisse, Feedback} (curriculum);
      \draw[line] (content) -- node[pos=0.55, right, xshift=8pt]{Materialien} (lms);
      \draw[line] (lernorte) -- node[pos=0.55, right, xshift=8pt, yshift=-2pt]{Praxisimpulse} (lms);
      \draw[dashedline] (lernorte) -- node[pos=0.7, below, xshift=6pt]{Koordination} (cohort);
      \draw[dashedline] (lernorte) -- node[pos=0.7, below, xshift=-6pt]{Anforderungen} (curriculum);
    \end{tikzpicture}%
  }
  \captionsetup{justification=justified,singlelinecheck=false}
  \caption{Systemisches Modell des eingesetzten Learning Management Systems mit Rückkopplung zwischen curricularen Handlungssituationen, LMS-Kern und kohortenspezifischen Ausbildungskursen.}
  \caption*{\footnotesize Dargestellt ist die Architektur des Systems als Kopplung von curricularer Struktur (Handlungssituationen), LMS-Kern und kohortenspezifischen Ausbildungskursen. Content (Materialien) und Lernorte (Praxisimpulse) wirken in den LMS-Kern ein; gestrichelte Relationen markieren Koordination und Anforderungen der Lernorte gegenüber Kurs- und Curricularebene.}
  \label{fig:modell_LMS}
\end{figure}

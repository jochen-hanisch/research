\begin{figure}[H]
  \centering
  % Shared TikZ styles for LMS-related schematic figures
% Intentionally light-weight: safe to \input{} inside figure files.
\usetikzlibrary{arrows.meta,backgrounds,calc,fit,positioning,shapes.geometric,shapes.misc}

% CI palette (defaults; overridden if already defined elsewhere)
\providecolor{ciPrimaryLine}{HTML}{1E5D8A}
\providecolor{ciSecondaryLine}{HTML}{4A403C}
\providecolor{ciAccent}{HTML}{89572A}
\providecolor{ciNeutral}{HTML}{595959}
\providecolor{ciGlint}{HTML}{E0E0E0}
\providecolor{ciNegative}{HTML}{892A5C}

\colorlet{lmsBlue}{ciPrimaryLine}
\colorlet{lmsPurple}{ciNegative}
\colorlet{lmsOrange}{ciAccent}
\colorlet{lmsGray}{ciNeutral}
\colorlet{lmsLight}{white}

\tikzset{
  lms/.style={
    font=\scriptsize,
    line width=0.55pt,
    draw=lmsGray,
    >=Latex,
  },
  lmsBox/.style={
    rectangle,
    rounded corners=6pt,
    draw=lmsGray,
    fill=lmsLight,
    align=center,
    inner sep=6pt,
    minimum height=10mm,
  },
  lmsLabel/.style={
    font=\scriptsize\bfseries,
    text=white,
    inner sep=6pt,
    rounded corners=6pt,
    align=left,
  },
  lmsArrow/.style={-{Latex[length=3mm,width=2mm]}, draw=lmsGray, line width=0.6pt},
  lmsArrowDashed/.style={lmsArrow, dashed},
  lmsDb/.style={
    cylinder,
    cylinder uses custom fill,
    cylinder body fill=lmsLight,
    cylinder end fill=white,
    shape border rotate=90,
    aspect=0.25,
    draw=lmsGray,
    minimum height=12mm,
    minimum width=12mm,
    align=center,
    inner sep=3pt,
  },
  lmsServer/.style={
    rectangle,
    rounded corners=2pt,
    draw=lmsGray,
    fill=lmsLight,
    minimum width=14mm,
    minimum height=18mm,
    align=center,
    inner sep=3pt,
  },
  lmsClient/.style={
    rectangle,
    rounded corners=2pt,
    draw=lmsGray,
    fill=white,
    minimum width=16mm,
    minimum height=11mm,
    align=center,
    inner sep=2pt,
  },
}

  \resizebox{\linewidth}{!}{%
    \begin{tikzpicture}[
      lms,
      node distance=1.4cm and 2.0cm
    ]
      \node[lmsBox, minimum width=3.4cm, minimum height=1.1cm] (curriculum) {Handlungssituationen\\(32 curriculare Einheiten)};
      \node[lmsBox, minimum width=3.4cm, minimum height=1.1cm, right=of curriculum, xshift=1.4cm] (lms) {LMS-Kern\\Kurse, Ressourcen, Aufgaben, Feedback};
      \node[lmsBox, minimum width=3.4cm, minimum height=1.1cm, right=of lms, xshift=1.4cm] (cohort) {Ausbildungskurse\\(3 Kohortenräume)};
      \node[lmsBox, minimum width=3.4cm, minimum height=1.1cm, above=of lms, yshift=0.6cm] (content) {Content\\Fachliteratur, Medien};
      \node[lmsBox, minimum width=3.4cm, minimum height=1.1cm, below=of lms, yshift=-0.6cm] (lernorte) {Lernorte\\Schule, Lehrrettungswache, Krankenhaus};

      \draw[lmsArrow] (curriculum) -- node[above, align=center]{curriculare Struktur} (lms);
      \draw[lmsArrow] (lms) -- node[above, align=center]{Umsetzung \& Steuerung} (cohort);
      \draw[lmsArrow] (curriculum) to[bend left=22] node[pos=0.55, above, yshift=4pt]{Aufgaben, Standards} (cohort);
      \draw[lmsArrow] (cohort) to[bend left=22] node[pos=0.45, below, yshift=-4pt]{Erkenntnisse, Feedback} (curriculum);
      \draw[lmsArrow] (content) -- node[pos=0.55, right, xshift=8pt]{Materialien} (lms);
      \draw[lmsArrow] (lernorte) -- node[pos=0.55, right, xshift=8pt, yshift=-2pt]{Praxisimpulse} (lms);
      \draw[lmsArrowDashed] (lernorte) -- node[pos=0.7, below, xshift=6pt]{Koordination} (cohort);
      \draw[lmsArrowDashed] (lernorte) -- node[pos=0.7, below, xshift=-6pt]{Anforderungen} (curriculum);
    \end{tikzpicture}%
  }
  \captionsetup{justification=justified,singlelinecheck=false}
  \caption{Systemisches Modell des eingesetzten Learning Management Systems mit Rückkopplung zwischen curricularen Handlungssituationen, LMS-Kern und kohortenspezifischen Ausbildungskursen.}
  \caption*{\footnotesize Dargestellt ist die Architektur des Systems als Kopplung von curricularer Struktur (Handlungssituationen), LMS-Kern und kohortenspezifischen Ausbildungskursen. Content (Materialien) und Lernorte (Praxisimpulse) wirken in den LMS-Kern ein; gestrichelte Relationen markieren Koordination und Anforderungen der Lernorte gegenüber Kurs- und Curricularebene.}
  \label{fig:modell_LMS}
\end{figure}

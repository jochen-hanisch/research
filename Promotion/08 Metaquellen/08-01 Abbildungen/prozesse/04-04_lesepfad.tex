\begin{figure}[H]
\centering
\begin{tikzpicture}[node distance=2.3cm and 2cm, on grid, auto,
  every node/.style={rectangle, draw, align=center, minimum height=1.4cm, text width=4.4cm},
  concept/.style={rectangle, draw=black, thick}
  ]

\node[concept] (paradigma) {4.1\\Erkenntnistheoretischer\\Rahmen};
\node[concept, below=of paradigma] (erhebung) {4.2\\Empirische\\Zugänge};
\node[concept, below=of erhebung] (analyse) {4.3\\Analytische\\Verfahren};
\node[concept, below=of analyse] (simulation) {4.4\\Systemische\\Modellprüfung};
\node[concept, below=of simulation] (reflexion) {4.5\\Methodenkritische\\Rückkopplung};

\draw[->, thick] (paradigma) -- (erhebung);
\draw[->, thick] (erhebung) -- (analyse);
\draw[->, thick] (analyse) -- (simulation);
\draw[->, thick] (simulation) -- (reflexion);

\draw[->, dashed, bend left=45] (reflexion) to (paradigma);
\draw[->, dashed, bend left=35] (simulation) to (analyse);

\end{tikzpicture}
\caption*{\footnotesize Zwei mögliche Lektürepfade durch Kapitel 4: Der lineare Pfad (schwarz, durchgezogen) folgt dem klassischen Forschungsprozess von der erkenntnistheoretischen Rahmung bis zur methodenkritischen Rückkopplung. Der explorative Pfad (grau, gestrichelt) betont zyklische Überprüfungen von Passung, Validität und Modellstimmigkeit.}
\caption{Strukturierte Lektürepfade durch Kapitel 4 – linear und explorativ.}
\label{fig:lesepfad-04}
\end{figure}

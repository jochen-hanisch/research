\begin{figure}[H]
  \centering
  % Shared TikZ styles for LMS-related schematic figures
% Intentionally light-weight: safe to \input{} inside figure files.
\usetikzlibrary{arrows.meta,backgrounds,calc,fit,positioning,shapes.geometric,shapes.misc}

% CI palette (defaults; overridden if already defined elsewhere)
\providecolor{ciPrimaryLine}{HTML}{1E5D8A}
\providecolor{ciSecondaryLine}{HTML}{4A403C}
\providecolor{ciAccent}{HTML}{89572A}
\providecolor{ciNeutral}{HTML}{595959}
\providecolor{ciGlint}{HTML}{E0E0E0}
\providecolor{ciNegative}{HTML}{892A5C}

\colorlet{lmsBlue}{ciPrimaryLine}
\colorlet{lmsPurple}{ciNegative}
\colorlet{lmsOrange}{ciAccent}
\colorlet{lmsGray}{ciNeutral}
\colorlet{lmsLight}{white}

\tikzset{
  lms/.style={
    font=\scriptsize,
    line width=0.55pt,
    draw=lmsGray,
    >=Latex,
  },
  lmsBox/.style={
    rectangle,
    rounded corners=6pt,
    draw=lmsGray,
    fill=lmsLight,
    align=center,
    inner sep=6pt,
    minimum height=10mm,
  },
  lmsLabel/.style={
    font=\scriptsize\bfseries,
    text=white,
    inner sep=6pt,
    rounded corners=6pt,
    align=left,
  },
  lmsArrow/.style={-{Latex[length=3mm,width=2mm]}, draw=lmsGray, line width=0.6pt},
  lmsArrowDashed/.style={lmsArrow, dashed},
  lmsDb/.style={
    cylinder,
    cylinder uses custom fill,
    cylinder body fill=lmsLight,
    cylinder end fill=white,
    shape border rotate=90,
    aspect=0.25,
    draw=lmsGray,
    minimum height=12mm,
    minimum width=12mm,
    align=center,
    inner sep=3pt,
  },
  lmsServer/.style={
    rectangle,
    rounded corners=2pt,
    draw=lmsGray,
    fill=lmsLight,
    minimum width=14mm,
    minimum height=18mm,
    align=center,
    inner sep=3pt,
  },
  lmsClient/.style={
    rectangle,
    rounded corners=2pt,
    draw=lmsGray,
    fill=white,
    minimum width=16mm,
    minimum height=11mm,
    align=center,
    inner sep=2pt,
  },
}

  \resizebox{\linewidth}{!}{%
    \begin{tikzpicture}[lms, node distance=12mm and 18mm]
      % Clients
      \node[lmsClient] (c1) {Client\\(Kurs)};
      \node[lmsClient, below=of c1] (c2) {Client\\(Listen)};
      \node[lmsClient, below=of c2] (c3) {Client\\(Auswertung)};

      % Server + DB
      \node[lmsServer, right=55mm of c2] (srv) {Web-\\Server};
      \node[lmsDb, right=28mm of srv] (db) {DB};

      % Labels
      \node[above=2mm of srv, font=\scriptsize\bfseries, text=lmsGray] {Applikation};
      \node[above=2mm of db, font=\scriptsize\bfseries, text=lmsGray] {Datenhaltung};

      % Flows (client <-> server)
      \draw[lmsArrow] (c1.east) -- node[above, sloped]{Daten abgeben} (srv.west);
      \draw[lmsArrow] (c2.east) -- node[above, sloped]{Daten filtern} (srv.west);
      \draw[lmsArrow] (c3.east) -- node[above, sloped]{Statistik erzeugen} (srv.west);

      % Server <-> DB
      \draw[lmsArrow] (srv.east) -- node[above]{Daten verwalten} (db.west);
      \draw[lmsArrowDashed] (db.west) -- node[below]{Antworten/Resultate} (srv.east);
    \end{tikzpicture}%
  }
  \captionsetup{justification=justified,singlelinecheck=false}
  \caption{User-Server-Interaktion.}
  \caption*{\footnotesize Schematisierte Interaktion zwischen Client, Webserver/Applikationsserver und Datenbank. Die Darstellung dient der Veranschaulichung der Log- und Nachweisfähigkeit auf Systemebene.}
  \label{fig:fg-user-server-interaktion}
\end{figure}


\begin{figure}[H]
\centering
\begin{tikzpicture}[node distance=2.3cm and 2cm, on grid, auto,
  every node/.style={rectangle, draw, align=center, minimum height=1.4cm, text width=4.4cm},
  concept/.style={rectangle, draw=black, thick}
  ]

\node[concept] (rueckbindung) {6.1\\Systemische Rückbindung\\der Ergebnisse};
\node[concept, below=of rueckbindung] (theorie) {6.2\\Theoriebezogene\\Modellverdichtung};
\node[concept, below=of theorie] (praxis) {6.3\\Gestalterische\\Konsequenzen};

\draw[->, thick] (rueckbindung) -- (theorie);
\draw[->, thick] (theorie) -- (praxis);

\draw[->, dashed, bend left=45] (praxis) to (rueckbindung);

\end{tikzpicture}
\caption*{\footnotesize Zwei Lektürepfade durch Kapitel 6: Der lineare Pfad (schwarz, durchgezogen) zeigt die argumentative Kette von Ergebnisrückbindung über Theoriemodifikation zur Gestaltung. Der explorative Pfad (grau, gestrichelt) betont die Rückkopplung gestalterischer Überlegungen auf theoretische Modelle und Forschungslogik.}
\caption{Strukturierte Lektürepfade durch Kapitel 6 – linear und explorativ.}
\label{fig:lesepfad-06}
\end{figure}

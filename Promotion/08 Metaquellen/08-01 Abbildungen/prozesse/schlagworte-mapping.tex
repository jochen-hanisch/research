\begin{figure}[H]
  \centering
  \begin{tikzpicture}[
    node distance=4mm and 10mm,
    font=\scriptsize,
    >=Latex,
    every node/.style={align=center},
    kw/.style={rectangle, draw, rounded corners, fill=blue!5, text width=6.4cm, inner sep=3pt},
    outbox/.style={rectangle, draw, rounded corners, fill=green!8, text width=4.4cm, inner sep=3pt},
    line/.style={->, thick},
    link/.style={<->, thick}
  ]

  \node[kw] (kw) {Schlagworte (Analyse 1.~Ordnung)\\
  \textit{Datenschutz und IT-Sicherheit; Krisenreaktion im Bildungsbereich; Lernsystemarchitektur; Bewertungsmethoden; Kollaboratives Lernen; Bildungstheorien; Forschungsans\"atze; Systemanpassung; Lehr- und Lerneffektivit\"at; Technologieintegration}};

  \node[outbox, right=12mm of kw] (folgen) {Folgen\\Wirkungsdimensionen\\St\"orungen/Nebenfolgen};
  \node[outbox, below=12mm of folgen] (impl) {Implikationen\\Organisation/Implementierung\\Technik/Daten\\Recht/Ethik\\Forschung/Evaluation};

  \draw[link] (kw) -- (folgen);

  \draw[line] (folgen) -- node[right, font=\footnotesize\itshape]{nachgelagert} (impl);
  \node[above=2mm of folgen, font=\footnotesize\itshape] {Ordnung der Schlagworte entlang der Indexstruktur};
  \end{tikzpicture}
  \captionsetup{justification=justified,singlelinecheck=false}
  \caption{Kopplung von Schlagworten, Folgen und Implikationen.}
  \caption*{\footnotesize Schematische Darstellung mit drei Ebenen: Schlagworte (Analyse 1. Ordnung) als Sammelkasten, daran gekoppelt die Folgen als Wirkungsdimensionen und Stoerungen/Nebenfolgen sowie darunter die Implikationen mit den Feldern Organisation/Implementierung, Technik/Daten, Recht/Ethik und Forschung/Evaluation.}
  \label{fig:schlagworte-mapping}
\end{figure}

\begin{figure}[H]
  \centering
  \begin{tikzpicture}[
    node distance=4mm and 8mm,
    font=\scriptsize,
    >=Latex,
    every node/.style={align=center},
    kw/.style={rectangle, draw, rounded corners, fill=blue!5, text width=3.0cm, inner sep=2pt},
    outbox/.style={rectangle, draw, rounded corners, fill=green!8, text width=4.4cm, inner sep=3pt},
    line/.style={->, thick}
  ]

  \node[kw] (kw1) {Datenschutz und IT-Sicherheit};
  \node[kw, below=of kw1] (kw2) {Krisenreaktion im Bildungsbereich};
  \node[kw, below=of kw2] (kw3) {Lernsystemarchitektur};
  \node[kw, below=of kw3] (kw4) {Bewertungsmethoden};
  \node[kw, below=of kw4] (kw5) {Kollaboratives Lernen};

  \node[kw, right=of kw1, xshift=2mm] (kw6) {Bildungstheorien};
  \node[kw, below=of kw6] (kw7) {Forschungsans\"atze};
  \node[kw, below=of kw7] (kw8) {Systemanpassung};
  \node[kw, below=of kw8] (kw9) {Lehr- und Lerneffektivit\"at};
  \node[kw, below=of kw9] (kw10) {Technologieintegration};

  \node[outbox, right=10mm of kw3] (folgen) {Folgen\\Wirkungsdimensionen\\St\"orungen/Nebenfolgen};
  \node[outbox, below=12mm of folgen] (impl) {Implikationen\\Organisation/Implementierung\\Technik/Daten\\Recht/Ethik\\Forschung/Evaluation};

  \draw[line] (kw1.east) -- (folgen.west);
  \draw[line] (kw2.east) -- (folgen.west);
  \draw[line] (kw3.east) -- (folgen.west);
  \draw[line] (kw4.east) -- (folgen.west);
  \draw[line] (kw5.east) -- (folgen.west);
  \draw[line] (kw6.east) -- (folgen.west);
  \draw[line] (kw7.east) -- (folgen.west);
  \draw[line] (kw8.east) -- (folgen.west);
  \draw[line] (kw9.east) -- (folgen.west);
  \draw[line] (kw10.east) -- (folgen.west);

  \draw[line] (folgen) -- node[right]{nachgelagert} (impl);
  \node[above=2mm of folgen] {Kopplung Schlagworte $\rightarrow$ Folgen};
  \end{tikzpicture}
  \captionsetup{justification=justified,singlelinecheck=false}
  \caption{Kopplung von Schlagworten, Folgen und Implikationen.}
  \caption*{\footnotesize Die Kopplungen sind nicht kausal gemeint. Schlagworte fungieren als strukturierende Begriffsanker. Implikationen sind den Folgen nachgelagert. Die Reihenfolge der Schlagworte folgt der Korrelationsstruktur der Indizes, vgl. Abb.~\ref{fig:A-kor-indizes}.}
  \label{fig:schlagworte-mapping}
\end{figure}
